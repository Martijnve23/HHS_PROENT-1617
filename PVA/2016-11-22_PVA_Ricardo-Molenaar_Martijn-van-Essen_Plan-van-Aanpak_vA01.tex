\documentclass[numbers=endperiod]{scrartcl}



%test

%\usepackage{fontspec}
%\setmainfont{Calibri}
\renewcommand{\familydefault}{\sfdefault}


\usepackage{geometry}
\geometry{
	a4paper,
	total={170mm,257mm},
	left=20mm,
	right=20mm,
	top=20mm,
	bottom=39mm
}

\usepackage{fancyhdr}
\usepackage{lastpage}
\usepackage{graphicx}
\usepackage{caption}

\usepackage[table]{xcolor}% http://ctan.org/pkg/xcolor
%\captionsetup{belowskip=1pt,aboveskip=0pt}


\graphicspath{ {./images/} }
\renewcommand{\figurename}{Figuur }
\renewcommand{\tablename}{Tabel }



\usepackage{xcolor}

\definecolor{hhs_theme_heading_1}{RGB}{26,104,169}
\definecolor{hhs_theme_heading_2}{RGB}{179,214,243}
\definecolor{hhs_theme_heading_3}{RGB}{170,183,52}

\addtokomafont{section}{\normalfont\Large}

\setkomafont{section}{\sectionheading}
\newcommand{\sectionheading}[1]{%
	\normalfont
	\Large%\sf\bf%
	\setlength{\fboxsep}{0cm}%already boxed
	\colorbox{hhs_theme_heading_1!80}{%
		\begin{minipage}{\linewidth}%
			\vspace*{6pt}%Space before
			\hspace*{4pt}
			\color{white}
			#1
			\vspace*{4pt}%Space after
		\end{minipage}%
}}

\setkomafont{subsection}{\subsectionheading}
\newcommand{\subsectionheading}[1]{%
	\vspace{-10pt}
	\normalfont
	\Large
	\setlength{\fboxsep}{0cm}%already boxed
	\colorbox{hhs_theme_heading_2!80}{%
		\begin{minipage}{\linewidth}%
			\vspace*{4pt}%Space before
			\hspace*{2pt}
			#1
			\vspace*{4pt}%Space after
		\end{minipage}%
}}

\setkomafont{subsubsection}{\subsubsectionheading}
\newcommand{\subsubsectionheading}[1]{%
	\normalfont
	\noindent
		\begin{minipage}{\linewidth}%
			\textcolor{hhs_theme_heading_1}{\rule{\textwidth}{1.5pt}}
			\vspace*{4pt}%Space before
			\hspace*{-29pt}
			#1
			\vspace*{2pt}%Space after
		\end{minipage}%
}
%end test



\DeclareGraphicsExtensions{.pdf,.png,.jpg}
\pagestyle{fancy}
%\fancyhf{}
%\rfoot{Pagina \thepage \hspace{1pt} van \pageref{LastPage}}

\setlength{\headheight}{35pt} 
\setlength{\footheight}{135pt} 


\fancyhead[L]{%\begin{minipage}[b]{0.65\linewidth}
	\includegraphics[scale=0.4]{Logo.png}
\fancyhead[C]{\vspace{-40pt}\textcolor{hhs_theme_heading_1}{\tiny In opdracht van: \\}\includegraphics[scale=0.14]{rijkswaterstaat.jpg}}}
\fancyhead[R]{\slshape \leftmark}

\fancyfoot[L]{ 
\tiny 2016-11-22\textunderscore PVA\textunderscore Ricardo-Molenaar\\\textunderscore Martijn-van-Essen\textunderscore Plan-van-Aanpak\textunderscore vA01
}
\fancyfoot[C]{	%\begin{flushleft}
				%\hspace{100pt}
				%\vspace*{0pt}
				\small Ricardo Molenaar \& Martijn van Essen\\
				PROENT\\
				V1.0 \textbar 18-11-2016
				%\end{flushleft}
			}	
\fancyfoot[R]{\vspace{0pt}\small \textbar Pagina \thepage \hspace{1pt} van \pageref{LastPage}\textbar}

\newcommand{\whitespace}{\vspace*{2 mm} \\}%Space between paragraphs
\newcommand{\moveup}{\vspace*{-7pt}}%Space between paragraphs



\renewcommand{\footrulewidth}{0.4pt}%

\title{Plan van Aanpak PROENT}
\date{18-11-2016}
\author{Martijn van Essen \& Ricardo Molenaar}

\usepackage{pdfpages}

\begin{document}
	%Title page
	\pagenumbering{gobble} %Turn off page numbers
	%\maketitle
	%\newgeometry{left=0cm,bottom=0cm}
	%\pagestyle{empty}
	%\vspace*{-3cm}
	%\includegraphics[scale=1]{cover.png}
	\includepdf{input}
	\newpage
	%\restoregeometry
	%End of title page
	
	%Versions page
	\pagenumbering{arabic} %Turn on page numbers
	\setcounter{secnumdepth}{0} %Turn off section numbering
	\section{Versiebeheer}
	
	\begin{center}
		\begin{tabular}{| p{4cm} | l | p{7cm} | l |}
			\hline
			
			\multicolumn{4}{|c|}{
				\cellcolor{hhs_theme_heading_2}
				Versiehistorie
			}  \\ \hline
			
			Versie 	& Datum 		& Wijzigingen 	& Auteur \\ \hline
			vA01 	& 22-11-2016 	& Alle wijzigingen 
			&Auteurs\\ \hline
		\end{tabular}
	\end{center}
	\newpage
	%End of Versions Page
	
	%Table of contents
	\renewcommand{\contentsname}{Inhoudsopgave} %Set table of contents section name
	\tableofcontents
	\newpage
	%End of Table of contents
	
	\setcounter{secnumdepth}{3}%Turn on section numbering

	\section{Projectachtergrond}
	In dit hoofdstuk worden de achtergronden van het project besproken. Hier wordt informatie gegeven over de achtergrond en algemene gegevens van het project.
	
	\subsection{Naam}
	Voor dit project is de naam PROENT - Windpark Borselen gekozen. PROENT is de projectnaam waar deze opdracht deel van uit maakt. Windpark Borssele II is de locatie waarvoor het ontwerp en beheers plan geschreven zal worden.
	\subsection{Opdrachtgever}
	De opdrachtgever voor dit project is Rijkswaterstaat. Zij verdelen de kavels waarop de windparken gebouwd zullen worden aan de hand van verschillende ontwerp en beheersplannen welke zij binnnen krijgen.
	\whitespace
	Rijkswaterstaat zal deze plannen beoordelen, maar zal verder geen actieve rol spelen in het aanleggen en beheren van het windpark.
	\subsection{Opdrachtnemer}
	De opdrachtnemer van dit project is ingenieursbureau Molenaar \& van Essen. Zij zijn verantwoordelijk voor het opleveren van een ontwerp en beheersplan bij Rijkswaterstaat.
	\subsection{Geschiedenis}
	In september 2013 zijn meer dan 40 organisaties akkoord gegegaan met het Energieakkoord -SOURCE-. In dit akkoord is afgesproken om in 2023 het aandeel van hernieuwbare energie met 16\% te verhogen.
	\whitespace
	Om dit plan te realiseren heeft Rijkswaterstaat een aantal kavels aangewezen voor de Nederlandse kust. Op deze kavels zullen windmolenparken gebouwd worden welke bijdragen aan het aandeel hernieuwbare energie.
	\subsection{Stakeholders}
	De opdrachtgever, Rijkswaterstaat, is bij dit project een grote belanghebbende. Zij zijn verantwoordelijk voor het bereiken van de eisen aan wind energie door de kavels op een goede manier te verdelen onder de bedrijven die deze willen exploiteren.
	\whitespace
	De Nederlandse overheid heeft ook belangen bij dit project. Door middel van dit project zijn zij in staat om aan de door de Europese Unie gestelde eisen aan duurzame energie te voldoen.
	\whitespace
	Verder heeft ook ingenieursbureau Molenaar \& van Essen. Zij kunnen in samenwerking met aannemer het windpark bouwen en vervolgens exploiteren.
	\subsection{Goedkeuring}
	De goedkeuring van het ontwerp en beheers plan zal door Rijkswaterstaat. Uit de verschillende inzendingen zullen zij het beste plan kiezen. Wanneer dit plan wordt goedgekeurd en uitgekozen zal de desbetreffende partij de rechten krijgen om dit plan uit te werken.
	\newpage

	\section{Projectresultaat}
	In dit hoofdstuk wordt het doel van het project beschreven met hierbij de resulaten die hiermee behaalt zullen worden.
	
	\subsection{Doelstelling}
	Om aan het energieakkoord te voldoen worden er windmolenparken op zee gebouwd. Het doel van dit project is om voor één van deze parken, namelijk Borssele II, een ontwerp en beheersplan te schrijven.
	\whitespace
	Aan de hand van dit plan kan het windmolenpark gebouwd en vervolgens beheerd worden.
	
	\subsection{Resultaat}	
	Het resultaat van dit project is een ontwerp en beheers plan. Dit plan zal worden ingediend bij Rijkswaterstaat waarna het beoordeeld zal worden.
	%\newpage

	\section{Projectactiviteiten}
	In dit hoofdstuk worden de activiteiten besproken welke in dit project plaats zullen vinden.
	\subsection{Documentatie}
	In dit project zal de volgende documentatie worden opgeleverd:
	\subsection{Bijeenkomsten}
		
	\subsection{Ontwerpen}
	
	\subsection{Beheersen}
				
	\newpage
	
	\section{Projectgrenzen}
	\newpage

	\section{Kwaliteit}
	\newpage

	\section{Projectorganisatie}
	Het gestelde project is aangenomen door ingenieursbureau Molenaar \& van Essen. Dit bedrijf bestaat uit
	twee werknemers welke beide aan dit project zullen werken. 
	\begin{table}[h]
	\caption{Projectorganisatie}\label{table:Projectorganisatie}
	
		\centering
		\begin{tabular}{ p{.3\textwidth} | p{.32\textwidth} | p{.3\textwidth} }
			Naam: 				& Studentnummer:& E-mailadres \\ \hline
			Ricardo Molenaar 	& 15087506	 	& R.Molenaar@student.hhs.nl \\
			Martijn van Essen 	& 15086135		& M.T.vanEssen@student.hhs.nl \\
		\end{tabular}
	
	\end{table}
	\newpage

	\section{Planning}
	In tabel~\ref{table:Planning} is een planning voor het verloop van dit project weergegeven.
	\begin{table}[h]
	\caption{Projectplanning}\label{table:Planning}
	\centering
	\begin{tabular}{| p{.3\textwidth} | p{.32\textwidth} | p{.3\textwidth} |}
			\hline \rowcolor{hhs_theme_heading_2}
			Item: 				& Geschat begin:& Deadline \\ \hline
			Plan van Aanpak 	& 17-11-2016 	& 28-11-2016 \\ \hline
			Projectplan		 	& 28-11-2016	& 16-1-2017 \\ \hline
			Procesverslag(?)	& 9-1-2017		& 16-1-2017 \\ \hline
			Eindassessment		& 17-1-2017		& 24-1-2017 \\ \hline
		\end{tabular}
	\end{table}
	\newpage


	
%	\begin{equation}
%	f(x) = x^2
%	\end{equation}
	
\end{document}